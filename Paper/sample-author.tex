%%
%% This is file `sample-manuscript.tex',
%% generated with the docstrip utility.
%%
%% The original source files were:
%%
%% samples.dtx  (with options: `manuscript')
%% 
%% IMPORTANT NOTICE:
%% 
%% For the copyright see the source file.
%% 
%% Any modified versions of this file must be renamed
%% with new filenames distinct from sample-manuscript.tex.
%% 
%% For distribution of the original source see the terms
%% for copying and modification in the file samples.dtx.
%% 
%% This generated file may be distributed as long as the
%% original source files, as listed above, are part of the
%% same distribution. (The sources need not necessarily be
%% in the same archive or directory.)
%%
%% Commands for TeXCount
%TC:macro \cite [option:text,text]
%TC:macro \citep [option:text,text]
%TC:macro \citet [option:text,text]
%TC:envir table 0 1
%TC:envir table* 0 1
%TC:envir tabular [ignore] word
%TC:envir displaymath 0 word
%TC:envir math 0 word
%TC:envir comment 0 0
%%
%%
%% The first command in your LaTeX source must be the \documentclass command.
%%%% Small single column format, used for CIE, CSUR, DTRAP, JACM, JDIQ, JEA, JERIC, JETC, PACMCGIT, TAAS, TACCESS, TACO, TALG, TALLIP (formerly TALIP), TCPS, TDSCI, TEAC, TECS, TELO, THRI, TIIS, TIOT, TISSEC, TIST, TKDD, TMIS, TOCE, TOCHI, TOCL, TOCS, TOCT, TODAES, TODS, TOIS, TOIT, TOMACS, TOMM (formerly TOMCCAP), TOMPECS, TOMS, TOPC, TOPLAS, TOPS, TOS, TOSEM, TOSN, TQC, TRETS, TSAS, TSC, TSLP, TWEB.
% \documentclass[acmsmall]{acmart}

%%%% Large single column format, used for IMWUT, JOCCH, PACMPL, POMACS, TAP, PACMHCI
% \documentclass[acmlarge,screen]{acmart}

%%%% Large double column format, used for TOG
% \documentclass[acmtog, authorversion]{acmart}

%%%% Generic manuscript mode, required for submission
%%%% and peer review
\documentclass[manuscript,screen,review]{acmart}
%% Fonts used in the template cannot be substituted; margin 
%% adjustments are not allowed.
%%
%% \BibTeX command to typeset BibTeX logo in the docs
\AtBeginDocument{%
  \providecommand\BibTeX{{%
    \normalfont B\kern-0.5em{\scshape i\kern-0.25em b}\kern-0.8em\TeX}}}

%% Rights management information.  This information is sent to you
%% when you complete the rights form.  These commands have SAMPLE
%% values in them; it is your responsibility as an author to replace
%% the commands and values with those provided to you when you
%% complete the rights form.
\setcopyright{acmcopyright}
\copyrightyear{2018}
\acmYear{2018}
\acmDOI{XXXXXXX.XXXXXXX}

%% These commands are for a PROCEEDINGS abstract or paper.
\acmConference[Conference acronym 'XX]{Make sure to enter the correct
  conference title from your rights confirmation emai}{June 03--05,
  2018}{Woodstock, NY}
%
%  Uncomment \acmBooktitle if th title of the proceedings is different
%  from ``Proceedings of ...''!
%
\acmBooktitle{Woodstock '18: ACM Symposium on Neural Gaze Detection,
 June 03--05, 2018, Woodstock, NY} 
\acmPrice{15.00}
\acmISBN{978-1-4503-XXXX-X/18/06}


%%
%% Submission ID.
%% Use this when submitting an article to a sponsored event. You'll
%% receive a unique submission ID from the organizers
%% of the event, and this ID should be used as the parameter to this command.
%%\acmSubmissionID{123-A56-BU3}

%%
%% For managing citations, it is recommended to use bibliography
%% files in BibTeX format.
%%
%% You can then either use BibTeX with the ACM-Reference-Format style,
%% or BibLaTeX with the acmnumeric or acmauthoryear sytles, that include
%% support for advanced citation of software artefact from the
%% biblatex-software package, also separately available on CTAN.
%%
%% Look at the sample-*-biblatex.tex files for templates showcasing
%% the biblatex styles.
%%

%%
%% The majority of ACM publications use numbered citations and
%% references.  The command \citestyle{authoryear} switches to the
%% "author year" style.
%%
%% If you are preparing content for an event
%% sponsored by ACM SIGGRAPH, you must use the "author year" style of
%% citations and references.
%% Uncommenting
%% the next command will enable that style.
%%\citestyle{acmauthoryear}

%%
%% end of the preamble, start of the body of the document source.
% My Packages
\usepackage{cleveref}
\usepackage{comment}
\usepackage{csvsimple}
\usepackage{datatool}
\usepackage{todonotes}[disable]
\usepackage{graphicx}
\usepackage{subcaption}
\newcommand{\todokdinline}[1]{\todo[color=red!20,inline]{{KD: \small #1}}}
\newcommand{\todokd}[1]{\todo[color=red!20]{{\small #1 -- Kasra}}}
% \newcommand{\todocainline}[1]{\todo[color=yellow!20,inline]{{CA: \small #1}}}
\newcommand{\todocm}[1]{\todo[color=green!40]{\small #1 -- Cynthia}}
\newcommand{\todocmi}[1]{\todo[inline,color=green!40]{\small #1 -- Cynthia}}
\newcommand{\todoff}[1]{\todo[color=blue!20]{\small #1 -- Frank}}
\newcommand{\todoffinline}[1]{\todo[inline,color=blue!20]{\small #1 -- Frank}}
\newcommand{\todoed}[1]{\todo[color=cyan!20]{\small #1 -- Ed}}

\newcommand{\CitationNeeded}[1]{{\textbf {\color{red}Cite #1}}}
\newcommand{\ProofRead}[1]{{\textbf {\color{red}Proofread please #1}}}
\newcommand{\Complete}[0]{{\textbf {\color{red}Complete it }}}
\newcommand{\Rephrase}[1]{{\textbf {\color{red}Rephrase please #1}}}
\newcommand{\TD}[1]{{\color{red}{\textbf{TODO}}} #1}
\newcommand{\OK}[1]{{\color{green}{\textbf{DONE}}} #1}
\newcommand{\DO}[1]{{\color{blue}{\textbf{PROG}}} #1}
\newcommand{\ours}{\textsc{EMMA}}
% \newcommand{\ours}{\textsc{EMMA}}
\newcommand{\geom}{\textsc{Geometric Alignment}}
\newcommand{\supcon}{\textsc{SupCon}}

%%%%%%%%%%%%%%%%%%%%
% EDITING LINK: 
% https://www.overleaf.com/3575751412sjsdmszkxsyh
%%%%%%%%%%%%%%%%%%%%


\begin{document}

%%
%% The "title" command has an optional parameter,
%% allowing the author to define a "short title" to be used in page headers.
\title{Multimodal Language Learning in the Face of Missing Modalities}
% \\ Extended Multimodal Alignment \\ Combining Geometric and classification ... }

%%
%% The "author" command and its associated commands are used to define
%% the authors and their affiliations.
%% Of note is the shared affiliation of the first two authors, and the
%% "authornote" and "authornotemark" commands
%% used to denote shared contribution to the research.

\author{Kasra Darvish}
% \authornote{Both authors contributed equally to this research.}
\email{kasradarvish@umbc.edu}
% \orcid{1234-5678-9012}
% \author{G.K.M. Tobin}
% \authornotemark[1]
% \email{webmaster@marysville-ohio.com}
\affiliation{%
  \institution{University of Maryland Baltimore County}
  % \streetaddress{P.O. Box 1212}
  \city{Baltimore}
  \state{MD}
  \country{USA}
  % \postcode{43017-6221}
}

\author{Edward Raff}
\affiliation{%
  \institution{University of Maryland Baltimore County \\
      Booz Allen Hamilton}
  % \streetaddress{1 Th{\o}rv{\"a}ld Circle}
  \city{Baltimore}
  \country{USA}}
\email{Raff\_Edward@bah.com}

\author{Francis Ferraro}
\affiliation{%
  \institution{University of Maryland Baltimore County}
  % \streetaddress{1 Th{\o}rv{\"a}ld Circle}
  \city{Baltimore}
  \country{USA}}
\email{ferraro@umbc.edu}

\author{Cynthia Matuszek}
\affiliation{%
  \institution{University of Maryland Baltimore County}
  \city{Baltimore}
  \country{USA}}
\email{cmat@umbc.edu}
      
%%
%% By default, the full list of authors will be used in the page
%% headers. Often, this list is too long, and will overlap
%% other information printed in the page headers. This command allows
%% the author to define a more concise list
%% of authors' names for this purpose.
\renewcommand{\shortauthors}{Trovato and Tobin, et al.}

%%
%% The abstract is a short summary of the work to be presented in the
%% article.

\begin{abstract}
% Grounded language understanding, in which natural language is used as a query against objects in a physical environment, allows a real-world, intuitive mechanism by which users can instruct physical agents to engage in tasks such as object retrieval. Visuolinguistic approaches to such object inference tasks typically involve training on large pools of image/text pairs and then using language to subselect elements of the sensed environment. However, physical agents such as robots typically have access to sensory and interactive modalities beyond vision, and learning from multiple modalities can improve performance on downstream tasks. In order to fully leverage multimodal training data while being robust to missing information, we propose a generalized distance-based loss function that can be extended to learn retrieval models that incorporate an arbitrary number of modalities. We demonstrate the usability of our model on a grounded language object retrieval scenario, where an intelligent agent has to select an object given an unconstrained language command. We leverage four modalities including vision, depth sensing, text, and speech, and we show that our model can outperform state-of-the-art contrastive models when modalities are ablated.

    % shorter version
    
    Our study is motivated by needs in robotics and computer-human interaction, where an agent has many sensors and thus modalities with which a human may interact, both to communicate a desired goal and for the agent to recognize a desired target object. For such problems, there has been little research on integrating more than two modalities. While there have been widely popular works based on cross-entropy, there is an entirely separate family of approaches based on explicit geometric alignment known as contrastive learning. However, to the best of our knowledge there has been no work on combining the two approaches for multimodal learning. We propose to combine both families of approaches and argue that the two are complementary. We introduce \textit{extended multimodal alignment (EMMA)}, a geometric (distance-based) method combined with a contrastive loss function. Our method can be used to learn retrieval models that incorporate an arbitrary number of views of a particular piece of data, compounded by the challenge of retrieval when a modality becomes unavailable. We demonstrate the usability of our model on a grounded language object retrieval scenario, where an intelligent agent has to select an object given an unconstrained language command. We leverage four modalities including vision, depth sensing, text, and speech, and we show that our model converges approximately 5 times faster than previous strong baselines, and out-performs or is strongly competitive with state-of-the-art contrastive learning. % (in 80\% less time).
    %when modalities are ablated. 
    %TODO: another baseline where we use cross-entropy loss only is missing I guess.
    The code is publicly available on GitHub and will be included for the camera-ready version (it is redacted for anonymity).
\end{abstract}

%%
%% The code below is generated by the tool at http://dl.acm.org/ccs.cfm.
%% Please copy and paste the code instead of the example below.
%%
\begin{CCSXML}
<ccs2012>
 <concept>
  <concept_id>10010520.10010553.10010562</concept_id>
  <concept_desc>Computer systems organization~Embedded systems</concept_desc>
  <concept_significance>500</concept_significance>
 </concept>
 <concept>
  <concept_id>10010520.10010575.10010755</concept_id>
  <concept_desc>Computer systems organization~Redundancy</concept_desc>
  <concept_significance>300</concept_significance>
 </concept>
 <concept>
  <concept_id>10010520.10010553.10010554</concept_id>
  <concept_desc>Computer systems organization~Robotics</concept_desc>
  <concept_significance>100</concept_significance>
 </concept>
 <concept>
  <concept_id>10003033.10003083.10003095</concept_id>
  <concept_desc>Networks~Network reliability</concept_desc>
  <concept_significance>100</concept_significance>
 </concept>
</ccs2012>
\end{CCSXML}

\ccsdesc[500]{Computer systems organization~Embedded systems}
\ccsdesc[300]{Computer systems organization~Redundancy}
\ccsdesc{Computer systems organization~Robotics}
\ccsdesc[100]{Networks~Network reliability}

%%
%% Keywords. The author(s) should pick words that accurately describe
%% the work being presented. Separate the keywords with commas.
\keywords{datasets, neural networks, gaze detection, text tagging}

%% A "teaser" image appears between the author and affiliation
%% information and the body of the document, and typically spans the
%% page.
\begin{comment}
\begin{teaserfigure}
  \includegraphics[width=\textwidth]{sampleteaser}
  \caption{Seattle Mariners at Spring Training, 2010.}
  \Description{Enjoying the baseball game from the third-base
  seats. Ichiro Suzuki preparing to bat.}
  \label{fig:teaser}
\end{teaserfigure}
\end{comment}

\received{20 February 2007}
\received[revised]{12 March 2009}
\received[accepted]{5 June 2009}

%%
%% This command processes the author and affiliation and title
%% information and builds the first part of the formatted document.
\maketitle






%%
%% The acknowledgments section is defined using the "acks" environment
%% (and NOT an unnumbered section). This ensures the proper
%% identification of the section in the article metadata, and the
%% consistent spelling of the heading.
\begin{acks}
To Robert, for the bagels and explaining CMYK and color spaces.
\end{acks}

%%
%% The next two lines define the bibliography style to be used, and
%% the bibliography file.
\bibliographystyle{ACM-Reference-Format}
\bibliography{references}

%%
%% If your work has an appendix, this is the place to put it.
\appendix
\label{sec:appendix}
% \todocmi{should we get rid of this appendix? I am thinking so.}
% 
\section{Loss Function Development Story}


Our proposed \geom{} loss function can be rewritten as shown in \cref{eq:objective}.
\begin{equation}\label{eq:objective}
\begin{split}
    \mathcal{L}  &= \sum_{i=1}^{M-1} \sum_{j=i+1}^{M} dist( z_{i}^{+} , z_{j}^{+}) 
     - dist( z_{i}^{+} , z_{j}^{-}) - dist( z_{i}^{-} , z_{j}^{+}) - \sum_{i=1}^{M} dist( z_{i}^{+} , z_{i}^{-} )
\end{split}
\end{equation}


We designed and experimented with alternative versions of our \geom{} loss function to analyze if they offer better results.
We start with a modification of the triplet loss function idea where we fix one modality as anchor and do not sample a positive instance from the same modality. 
Na\"ively, to extend triplet loss to work with arbitrary number of modalities, we can use two anchors (e.g., text and speech), which results in $2(M-2)$ triplet losses as formulated in~\cref{eq:objective-two-anchors}. 

% \todoffinline{in this equation block, you're writing things like $|| z_{t}^{+} - z_{m}^{+} ||$. Do you actually mean norm, or would any old scoring function work?}

\begin{equation}
\label{eq:objective-two-anchors}
\begin{split}
    \mathcal{L}  &= \sum_{m=1}^{M-2} dist( z_{t}^{+} , z_{m}^{+} ) - dist( z_{t}^{+} , z_{m}^{-} ) + dist( z_{s}^{+} , z_{m}^{+} ) - dist( z_{s}^{+} , z_{m}^{-} ) \\
    &= \sum_{m=1}^{M-2} \cos(z_{t}^{+} ,z_{m}^{-}) - \cos(z_{t}^{+}, z_{m}^{+}) + \cos(z_{s}^{+} ,z_{m}^{-}) - \cos(z_{s}^{+}, z_{m}^{+})
\end{split}
\end{equation}
where $M$ is the number of modalities, $t$ represents text modality, $s$ represents speech modality, and the superscripts $+$ and $-$ represent positive and negative objects, respectively.
The $\cos(\cdot)$ function is a measure of similarity, not distance, and that is why the signs are reversed. To measure the distance, we use cosine distance.
In order to use cosine \textit{distance}, we have to subtract the cosine of the \textit{angle} between two embeddings (which represents similarity) from 1: $1 - \cos(e_1, e_2)$.

However, this approach performs poorly when one of the anchor modalities is ablated; since there is no explicit minimization between the two anchors, text (t) and speech (s) do not necessarily map closely to each other, such that only one of the two anchors is actually learned. 





An alternative is to apply triplet loss $M-1$ times with a single fixed anchor (text in our case) and $M-1$ `target' modalities.  In this case, the negative anchor point is disregarded. The positive anchor is used as an anchor for all $M-1$ triplet losses, and for each of those triplet losses the positive and negative points are simply selected from the corresponding modalities. This can be derived from \cref{eq:full-emma} by fixing one modality as anchor, and excluding the distance between the positive and negative anchors or $\cos(z_{a}^{+}, z_{a}^{-})$.
\Cref{eq:objective-simple-mma} formulates this idea.

\begin{equation}
\label{eq:objective-simple-mma}
\begin{split}
    \mathcal{L}  &= \sum_{m=1}^{M-1} dist( z_{a}^{+} , z_{m}^{+} ) - dist( z_{a}^{+} , z_{m}^{-} ) \\
    &= \sum_{m=1}^{M-1} 1 - \cos(z_{a}^{+}, z_{m}^{+}) - (1 - \cos(z_{a}^{+}, z_{m}^{-})) \\
    &= \sum_{m=1}^{M-1} \cos(z_{a}^{+} ,z_{m}^{-}) - \cos(z_{a}^{+}, z_{m}^{+})
\end{split}
\end{equation}
Here, subscript $a$ represents the \textit{anchor} modality, $M$ is the number of modalities, superscript $+$ represents the positive objects, and superscript $-$ represents the negative objects.

However, this approach fails to take into account the dissimilarity of the positive and negative anchor points. This is an important piece of information that \cref{eq:objective-simple-mma} ignores. Although this information is implicitly captured when we maximize the distance between positive anchor and other negative modalities, when the negative data point becomes a positive example itself, the anchor and those points are forced to be closer to each other which can happen in three ways: either moving anchor closer to those points, moving those points closer to anchor, or move both somewhere in between. Therefore, in the last two cases we do not have explicit direct control over the distance between these points and previous positive points. Our experimental results confirm this hypothesis.

Therefore, we add an extra term which is responsible for maximizing the distance between positive and negative instances of the anchor modality, because a margin between positive and negative points from other modalities (excluding anchor) is enforced. This can be derived from \cref{eq:full-emma} by fixing one modality as anchor.
% The first dashed line from the top in \cref{fig:emma-loss} is this extra term that captures the dissimilarity between positive and negative text datapoints.
We formulate this method in \cref{eq:objective-explicit-fixed-anchor}, which is generalized to an arbitrary number of modalities.

\begin{equation}
\label{eq:objective-explicit-fixed-anchor}
\begin{split}
    \mathcal{L}  %&= %\sum_{m=1}^{M-1} || z_{a}^{+} - z_{m}^{+} || - || z_{a}^{+} - z_{m}^{-} ||  - || z_{a}^{+} - z_{a}^{-} || \\
    &= \max(\cos(z_{a}^{+}, z_{a}^{-}) + \alpha_a, 0) + \sum_{m=1}^{M-1} \max \left(\cos(z_{a}^{+} ,z_{m}^{-}) - \cos(z_{a}^{+}, z_{m}^{+}) + \alpha_m, 0 \right) \\  
    &= \sum_{m=1}^{M}  \cos(z_{a}^{+} ,z_{m}^{-}) - \sum_{m=1}^{M-1} \cos(z_{a}^{+}, z_{m}^{+})
\end{split}
\end{equation}


In \cref{eq:objective-explicit-fixed-anchor}, $a$ represents the \textit{anchor} modality, $M$ is the number of modalities, the superscripts $+$ and $-$ represent positive and negative objects, $\alpha_m$ represents enforced margin for each modality which we set to 0.4 for all modalities without tuning, and $z$ is the embedding we get by applying a mapping function $f$, which in our case is a neural network on our input data.
In other words, $z_m = f_m(x_m)$, where each modality $m$ has a specific model $f_m$ that is different from the models for other modalities. These models do not share their weights.

Since the downstream task we are considering in the grounded language learning domain is to predict/retrieve a desired object among multiple objects given a natural language description, it makes sense to train the model using language as an anchor, and in fact anchoring learning around language outperforms anchoring on RGB. However, our proposed \geom{} loss function formulated in \cref{eq:full-emma} where we treat all modalities as anchor once still results in a better performance.


% Triplet loss cannot be used for more than 2 modalities. Some previous work has concatenated RGB and depth embeddings to create a single ``vision'' embedding for learning~\cite{triplet_loss_2021_CVPR}, but they cannot handle RGB or depth sensor ablation during test. Since our method can handle any number of modalities, we can handle such a case by treating depth as a separate modality.
% Moreover, when using triplet loss with different anchors per batch, the batch size has to be 1 which makes training very slow.

% This approach is most similar to that of supervised contrastive learning~\cite{NEURIPS2020_supervised_contrastive}, but outperforms both that and regular contrastive learning~\cite{chen2020simple} methods.
% Especially when modalities are ablated, most notably written language.

% We find that the use of instance level losses grounded to one key modality (text) is sufficient to encourage a representation that satisfies our goals. 






\subsection{Extended Triplet}
Using an extended version of triplet loss to consider all pairwise connections which is formulated in \cref{eq:extended-triplet}, results in a worse performance.

\begin{equation}\label{eq:extended-triplet}
    L = \sum_{m_1=1}^{M} \left[ \sum_{m_2=m_1+1}^{M} \left[ \text{triplet}(z_{m_1}^{+}, z_{m_2}^{+}, z_{m_2}^{-}) + \text{triplet}(z_{m_1}^{-}, z_{m_2}^{-}, z_{m_2}^{+}) \right] + \cos(z_{m_1}^{+}, z_{m_1}^{-}) -1 + margin \right]
\end{equation}

% Didn't change much when we drop text: Full eMMA when we have similar objects 

% Did not work: Run the version with all connections (including among negatives) without triplet and just by using cosine distance between pair of embeddings



\subsection{Smaller Batch Sizes And Pulling Among Negatives}

Reducing the batch size from 64 to 16 and 8 (for both \ours{} and \supcon{}) increases the gap between our model and the baseline in cases where our model beats the baseline (all modalities, dropping speech), but baseline is still better than ours when we drop the text modality.

In batch size of 2 the trend is still the same, but when we drop the text modality, the MRR graph is flat at 0.4. This is still better than the worst and random cases, but it implies that the model does not learn anything with regards to speech.
To remedy this, we add connections among the modalities of the negative instance as formulated in \cref{eq:full-emma-pull-negatives} and as shown in \cref{fig:full-emma-pull-negatives}, and that solves the flatline problem when we drop the text modality; however, when we have all modalities it performs worse than \ours{}.

\begin{equation}\label{eq:full-emma-pull-negatives}
\begin{split}
    L = & \sum_{m_1=1}^{M} \left[ \sum_{m_2=1}^{M} \left[ \max (\cos(z_{m_1}^{+}, z_{m_2}^{-}) -1 + margin,0) \right] + \sum_{m_3=m_1+1}^{M} \left[ \max(1 - \cos(z_{m_1}^{+},z_{m_3}^{+}),0) \right] \right. \\
    & \left. + \sum_{m_3=m_1+1}^{M} \left[ \max(1 - \cos(z_{m_1}^{-},z_{m_3}^{-}),0) \right] \right]
\end{split}
\end{equation}

\begin{figure*}[tb]
\centering
\includegraphics[width=1\columnwidth]{Figures/unused/full-emma-pull-negatives.pdf}
\caption{A high-level prototype of our approach and the distances used when we have add pulling among negative modalities.}
\label{fig:full-emma-pull-negatives}
\end{figure*}



\subsection{Margin}
% \todokdinline{Try different margins of 0.6, 1.0, and if it improved the results we can try to learn it as a parameter}
% \todokdinline{Should we remove this section?}
We experimented with two different margins values: a margin of 0.6 and a margin of 1.0. Using a margin of 0.6 does not change the performance much. However, it converges even faster. Using a margin of 1.0 (this is essentially just cosine similarity because we have $cos -1 + \text{margin}$) results in a worse performance when we have all modalities. 
% The MRR is 0.8995 when margin is 1, while with a margin of 0.6 the MRR is 0.9176, and with a margin of 0.4 the MRR is 0.9198. If we drop the text modality, the MRR is 0.7923, 0.7913, and 0.6918 for margins of 0.4, 0.6, and 1.0 respectively.





\section{Results}
We find that using each modality as anchor results in a better performance compared to when we fix one modality as anchor. This difference is more significant when we test with all modalities. 
% 
While the performance of \supcon{} deteriorates as the batch size is reduced, our \geom{} stays the same when we have all modalities or when we drop speech. However, when we drop text, EMMA deteriorates with decreasing batch size, and with a batch size of 2 without pulling negatives, it flatlines. In this case (when batch size is 2 and text is dropped), pulling among negatives is important and results in a better performance.







\end{document}
\endinput
%%
%% End of file `sample-authordraft.tex'.
